\documentclass[11pt]{article}
\usepackage[utf8]{inputenc}
\usepackage[T1]{fontenc}
\usepackage[french]{babel}
\usepackage{geometry}
\usepackage{graphicx}
\usepackage{booktabs}
\usepackage{amsmath}
\usepackage{caption}
\usepackage{float}
\usepackage{fancyhdr}
\usepackage{lastpage}
\usepackage{xcolor}
\usepackage{enumitem}
\usepackage{array}
\usepackage{amssymb}

#A supprimer pour les figures
\setkeys{Gin}{draft}

\graphicspath{{figures/}}

\geometry{margin=2cm}
\pagestyle{fancy}
\fancyhead[L]{Projet NSGL}
\fancyhead[R]{Tanguy CESAR}
\fancyfoot[C]{\thepage/\pageref{LastPage}}
\renewcommand{\headrulewidth}{0.5pt}
\renewcommand{\footrulewidth}{0.5pt}
\setlength{\headheight}{20pt}
\setlist[itemize]{leftmargin=*,labelsep=1em}

\begin{document}

\begin{center}
    {\Large \textbf{Projet NSGL — Network Science and Graph Learning}}\\[0.3cm]
    Tanguy CESAR
\end{center}

\vspace{0.5cm}

%%%%%%%%%%%%%%%%%%%%%%%%%%%%%%%%%%%%%%%%%%%%%%%%%%%%%%%%%%%%
\section*{Introduction}

Ce projet vise à analyser des réseaux sociaux issus du jeu de données \textit{Facebook100}, en mobilisant des outils classiques de \textbf{network science} ainsi que des méthodes de \textbf{graph learning}.  
Les expériences portent sur l’analyse structurelle des graphes, l’étude de l’homophilie, la prédiction de liens, la propagation de labels et la détection de communautés.

%%%%%%%%%%%%%%%%%%%%%%%%%%%%%%%%%%%%%%%%%%%%%%%%%%%%%%%%%%%%
\section*{Question 2 — Analyse de réseaux sociaux}

Les réseaux étudiés dans cette partie sont ceux de \textit{Caltech}, \textit{MIT} et \textit{Johns Hopkins}.  
Pour chacun, nous considérons la plus grande composante connexe du graphe.

\subsection*{Distribution des degrés}

\begin{figure}[H]
    \centering
    \includegraphics[width=0.48\textwidth]{deg_hist_caltech.png}
    \includegraphics[width=0.48\textwidth]{deg_ccdf_caltech.png}
    \caption{Histogramme et CCDF des degrés — Caltech}
\end{figure}

Les distributions de degrés présentent une forte hétérogénéité, avec une majorité de sommets faiblement connectés et quelques nœuds très centraux. La représentation en échelle log-log met en évidence une queue lourde, typique des réseaux sociaux.

\subsection*{Clustering et densité}

\begin{table}[H]
\centering
\begin{tabular}{lcccc}
\toprule
Réseau & $n$ & Densité & Clustering global & Clustering moyen \\
\midrule
Caltech & & & & \\
MIT & & & & \\
Johns Hopkins & & & & \\
\bottomrule
\end{tabular}
\caption{Statistiques globales des réseaux}
\end{table}

Les graphes sont globalement peu denses mais présentent un clustering élevé, indiquant une forte tendance à la formation de triangles et de groupes locaux.

\subsection*{Lien degré – clustering local}

\begin{figure}[H]
    \centering
    \includegraphics[width=0.55\textwidth]{deg_vs_clust.png}
    \caption{Clustering local en fonction du degré}
\end{figure}

On observe une relation décroissante entre le degré et le clustering local, traduisant le fait que les nœuds très connectés relient souvent des communautés différentes.

%%%%%%%%%%%%%%%%%%%%%%%%%%%%%%%%%%%%%%%%%%%%%%%%%%%%%%%%%%%%
\section*{Question 3 — Assortativité et homophilie}

Nous mesurons l’assortativité des graphes selon différents attributs : statut, spécialité (major), résidence (dorm), genre et degré.

\begin{figure}[H]
    \centering
    \includegraphics[width=0.55\textwidth]{assortativity_vs_size.png}
    \caption{Assortativité en fonction de la taille du réseau}
\end{figure}

Les résultats montrent une forte homophilie pour les attributs sociaux (dorm, major), tandis que l’assortativité de degré est généralement faible ou légèrement négative.

%%%%%%%%%%%%%%%%%%%%%%%%%%%%%%%%%%%%%%%%%%%%%%%%%%%%%%%%%%%%
\section*{Question 4 — Prédiction de liens}

Nous implémentons trois métriques classiques :
\begin{itemize}
    \item Common Neighbors
    \item Jaccard
    \item Adamic-Adar
\end{itemize}

Une fraction $f$ des arêtes est supprimée aléatoirement, puis les liens manquants sont prédits à partir des scores.

\begin{figure}[H]
    \centering
    \includegraphics[width=0.6\textwidth]{precision_recall.png}
    \caption{Précision et rappel en fonction de $k$}
\end{figure}

Les résultats montrent que les méthodes basées sur les voisins communs sont efficaces pour les petites valeurs de $k$, Adamic-Adar offrant généralement les meilleures performances.

%%%%%%%%%%%%%%%%%%%%%%%%%%%%%%%%%%%%%%%%%%%%%%%%%%%%%%%%%%%%
\section*{Question 5 — Propagation de labels}

Nous appliquons un algorithme de propagation de labels pour prédire des attributs manquants (dorm, major, gender).  
Une fraction de 10\%, 20\% et 30\% des labels est masquée aléatoirement.

\begin{table}[H]
\centering
\begin{tabular}{lccc}
\toprule
Attribut & Accuracy & F1-macro & MAE \\
\midrule
Dorm & & & \\
Major & & & \\
Gender & & & \\
\bottomrule
\end{tabular}
\caption{Performances de la propagation de labels}
\end{table}

La résidence (\textit{dorm}) est généralement mieux prédite, ce qui s’explique par sa forte corrélation avec la structure communautaire du graphe.

%%%%%%%%%%%%%%%%%%%%%%%%%%%%%%%%%%%%%%%%%%%%%%%%%%%%%%%%%%%%
\section*{Question 6 — Détection de communautés}

\textbf{Question de recherche :}  
Les communautés détectées correspondent-elles principalement aux résidences étudiantes ?

Nous utilisons les algorithmes de Louvain et de maximisation gloutonne de la modularité.

\begin{table}[H]
\centering
\begin{tabular}{lccc}
\toprule
Méthode & Attribut & NMI & ARI \\
\midrule
Louvain & Dorm & & \\
Greedy & Dorm & & \\
\bottomrule
\end{tabular}
\caption{Comparaison communautés / attributs}
\end{table}

Les scores élevés de NMI et ARI pour l’attribut \textit{dorm} confirment que la structure communautaire reflète largement l’organisation résidentielle.

%%%%%%%%%%%%%%%%%%%%%%%%%%%%%%%%%%%%%%%%%%%%%%%%%%%%%%%%%%%%
\section*{Conclusion}

Ce projet met en évidence les propriétés classiques des réseaux sociaux universitaires : hétérogénéité des degrés, fort clustering, homophilie marquée et structure communautaire significative.  
Les méthodes simples de graph learning se révèlent efficaces, tant pour la prédiction de liens que pour la récupération d’attributs manquants.

\end{document}
